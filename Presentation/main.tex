%!TEX program = xelatex
% \documentclass{beamer}
\documentclass[aspectratio=169]{beamer}

% \usepackage[english]{babel}

% \usepackage{graphicx,hyperref,url, materialbeamer}
% \usepackage{braket}
% \usepackage{euler}
% \usepackage{listings}

\usepackage{materialbeamer}

% \graphicspath{ {../Images/} }
\setbeamercovered{transparent}
\lstdefinestyle{customsql}{
  belowcaptionskip=1\baselineskip,
  breaklines=true,
  xleftmargin=\parindent,
  language=SQL,
  showstringspaces=false,
  basicstyle=\footnotesize\ttfamily,
  keywordstyle=\bfseries\color{green!40!black},
  commentstyle=\itshape\color{purple!40!black},
  identifierstyle=\color{blue},
  stringstyle=\color{orange},
}
\lstset{escapechar=@,style=customsql}
\usefonttheme{professionalfonts} % using non standard fonts for beamer

% The title of the presentation:
%  - first a short version which is visible at the bottom of each slide;
%  - second the full title shown on the title slide;
% Todo: Fill your thesis title
\title[Diploma Thesis]{Example Thesis Title}

% Optional: a subtitle to be displayed on the title slide
\subtitle{Diploma Thesis}

% The author(s) of the presentation:
%  - again first a short version to be displayed at the bottom;
%  - next the full list of authors, which may include contact information;
% \author[L. Lancia \& G. Salillari]{L. Lancia, G. Salillari}
% Todo: Fill you full name
\author{Surname Name}

% Todo: Position logo in respect to your title
% set the y parameter in the \put(175, y) (in this example it is set 30)
\titlegraphic{
  \begin{picture}(0,0)
    \put(175,30){\makebox(0,0)[rt]{\includegraphics[width=3cm]{../Images/TUC_logo_circle.png}}}
  \end{picture}
}

% The institute:
%  - to start the name of the university as displayed on the top of each slide
%    this can be adjusted such that you can also create a Dutch version
%  - next the institute information as displayed on the title slide
\institute[Technical University of Crete]{Electrical \& Computer Engineering School\\Technical University of Crete}

% Add a date and possibly the name of the event to the slides
%  - again first a short version to be shown at the bottom of each slide
%  - second the full date and event name for the title slide
\date[\today]{\today}

% \providecommand{\di}{\mathop{}\!\mathrm{d}}
% \providecommand*{\der}[3][]{\frac{d\if?#1?\else^{#1}\fi#2}{d #3\if?#1?\else^{#1}\fi}}
%  \providecommand*{\pder}[3][]{%
%     \frac{\partial\if?#1?\else^{#1}\fi#2}{\partial #3\if?#1?\else^{#1}\fi}%
%   }
\begin{document}

\begin{frame}
	\titlepage
\end{frame}

\begin{frame}{Table of Contents}
	\tableofcontents
\end{frame}

\chapter{Introduction}
\label{Chapter-Introduction}

\section{Motivation}
\section{Scientific Contributions}
\section{Thesis Outline}
% Todo: Fill chapter descriptions
\begin{itemize}
	\item \textbf{Chapter 2 - Theoretical Background:} Chapter 2 description
	\item \textbf{Chapter 3 - Related Work:} Chapter 3 description
	\item \textbf{Chapter 4 - Robustness Analysis:} Chapter 4 description
	\item \textbf{Chapter 5 - FPGA Implementation:} Chapter 5 description
	\item \textbf{Chapter 6 - Results:} Chapter 6 description
	\item \textbf{Chapter 7 - Conclusions and Related Work:} Chapter 7 description
\end{itemize}

\setlength{\parskip}{\baselineskip}
\section{Theoretical Background}

\begin{frame}
	\huge Theoretical Background
\end{frame}

% Todo: Edit to your liking
\begin{frame}{Some Title}
	\begin{itemize}
		\item Some item 1
		\item Some item 2
		\item Some item 3
	\end{itemize}
\end{frame}

\chapter{Related Work}
\label{Chapter-Related-Work}

% Todo: edit to your liking
\section{Related work A}
\section{Related work B}

\section{The FPGA Perspective}
\section{Thesis Approach}

\chapter{Robustness Analysis}
\label{Chapter-Robustness-Analysis}

% Todo: edit to your liking
\section{Experiment A}
\section{Experiment B}

\setlength{\parskip}{\baselineskip}
\section{Architecture Design}

\begin{frame}
	\huge Architecture Design
\end{frame}

% Todo: Edit to your liking
\begin{frame}{Some Title}
	\begin{itemize}
		\item Some item 1
		\item Some item 2
		\item Some item 3
	\end{itemize}
\end{frame}

\setlength{\parskip}{\baselineskip}
\section{FPGA Implementation}

\begin{frame}
    \huge FPGA Implementation
\end{frame}

% Use as a reference - Edit to your liking

% Todo: Replace with your platform
\begin{frame}{Xilinx ZCU102 Evaluation Kit}
	\centering
	\includegraphics[width=0.65\textwidth]{../Images/Hardware/ZCU102-board-overview.jpg}\\
\end{frame}

% Todo: Replace with your toolchain
\begin{frame}{Tools Used: Xilinx Vivado HLS}
	\begin{minipage}{0.6\textwidth}
		\centering
		\includegraphics[width=0.8\textwidth]{../Images/Platform/Vivado-HLS.png}\\
	\end{minipage}%
	\begin{minipage}{0.4\textwidth}
		\begin{itemize}
			\item Now Vitis HLS
			\item High-level design using C/C++, SystemC, OpenCL
			\item Generates VHDL \& Verilog HDL designs
			\item Directives
			\item C/C++ testbench
			\item C/RTL Cosimulation
			\item Synthesis Report
		\end{itemize}
	\end{minipage}
\end{frame}

\begin{frame}{Tools Used: Xilinx Vivado IDE}
	\begin{minipage}{0.6\textwidth}
		\centering
		\includegraphics[width=0.9\textwidth]{../Images/Platform/Vivado-IDE.png}\\
	\end{minipage}%
	\begin{minipage}{0.4\textwidth}
		\begin{itemize}
			\item VHDL \& Verilog
			\item IP Integrator Tool
			\item Vivado HLS RTL designs
			\item Synthesis, Implementation \& Download RTL designs
			\item RTL Simulators \& Integrated Logic Analyzer IPs
		\end{itemize}
	\end{minipage}
\end{frame}

\begin{frame}{Tools Used: Xilinx SDK/Vitis IDE}
	\begin{minipage}{0.6\textwidth}
		\centering
		\includegraphics[width=\textwidth]{../Images/Platform/Vitis-IDE.png}\\
	\end{minipage}%
	\begin{minipage}{0.4\textwidth}
		\begin{itemize}
			\item Vitis IDE  integrates SDK, SDAccel, SDSoC tools
			\item C/C++ IDE
			\item Application development for PS part
			\item PetaLinux \& FreeRTOS
			\item Download bitstreams
			\item Debugging tools
		\end{itemize}
	\end{minipage}
\end{frame}

\setlength{\parskip}{\baselineskip}
\section{Results}

\begin{frame}
	\huge Results
\end{frame}

% Todo: Fill information
\begin{frame}{Compared Platforms: CPU}
	\center{\large{Some CPU model}}
	\begin{table}[H]
		\centering
		\begin{tabular}{ll}
			\toprule
			\textbf{Cores / Threads}      & x/x      \\
			\textbf{Max Turbo Frequency}  & x GHz   \\
			\textbf{TDP}                  & x W      \\
			\textbf{Max Memory Bandwidth} & x GB/s \\
			\textbf{Lithography}          & x nm     \\
			\bottomrule
		\end{tabular}
	\end{table}
\end{frame}

% Todo: Fill information
\begin{frame}{Compared Platforms: GPU}
	\center{\large{Some GPU model}}
	\begin{table}[H]
		\centering
		\begin{tabular}{ll}
			\toprule
			\textbf{CUDA Cores}        & x      \\
			\textbf{Tensor Cores}      & x        \\
			\textbf{GPU Memory}        & x GB GDDR6 \\
			\textbf{Boost Clock}       & x MHz  \\
			\textbf{Memory Interface}  & x-bit   \\
			\textbf{Memory Bandwidth}  & x GB/s   \\
			\textbf{Power Consumption} & x W      \\
			\bottomrule
		\end{tabular}
	\end{table}
\end{frame}

% Todo: Fill information
\begin{frame}{Compared Platforms: FPGA}
	\center{\large{Some FPGA Platform}}
	\begin{table}[H]
		\centering
		\begin{tabular}{ll}
			\toprule
			\textbf{PL/DSP Clock Frequency} & x/x MHz \\
			\textbf{LUT Usage}              & x\%      \\
			% 		\textbf{LUTRAM Usage} & -\\
			\textbf{FF Usage}               & x\%     \\
			\textbf{BRAM Usage}             & x\%     \\
			\textbf{DSP Usage}              & x\%     \\
			% 		\textbf{BUFG Usage} & -\\
			\bottomrule
		\end{tabular}
	\end{table}
\end{frame}

% Todo: Fill information
\begin{frame}{Compared Platforms: FPGA}
	\center{\large{Proposed Platform}}
	\begin{table}[H]
		\centering
		\begin{tabular}{ll}
			\toprule
			\textbf{Clock Frequency (MHz)} & xMHz \\
			\textbf{LUT Usage}             & x\% \\
			\textbf{LUTRAM Usage}          & x\% \\
			\textbf{FF Usage}              & x\% \\
			\textbf{BRAM Usage}            & x\% \\
			\textbf{DSP Usage}             & x\%  \\
			% \textbf{BUFG (\%)}             & x\% \\
			\bottomrule
		\end{tabular}
	\end{table}
\end{frame}

% Todo: Fill configuration information
\begin{frame}{CPU \& GPU Performance}
	\begin{itemize}
		\item Config 1
		\item Config 2
		\item Config 3
	\end{itemize}
\end{frame}

% Todo: Replace graph
\begin{frame}{CPU \& GPU Performance: Latency}
	\centering
	\includegraphics[width=0.7\textwidth]{../Images/Results/CPU-GPU-Inference-Latency.png}\\
\end{frame}

% Todo: Replace graph
\begin{frame}{CPU \& GPU Performance: Throughput}
	\centering
	\includegraphics[width=0.7\textwidth]{../Images/Results/CPU-GPU-Inference-Throughput.png}\\
\end{frame}

% Todo: Fill information
\begin{frame}{Final Performance}
	\begin{table}[H]
		\centering
		\begin{tabular}{l|l|l|l|l}
			\toprule
			                                    & \textbf{CPU} & \textbf{GPU} & \textbf{CHaiDNN} & \textbf{Proposed Platform} \\
			\midrule
			\textbf{Clock Frequency (MHz)}      & x				& x				& x					& x							\\
			\textbf{Throughput (Images/s)}      & x				& x				& x					& x							\\
			\textbf{Throughput Speedup}         & 100\%			& x\%			& x\%				& x\%						\\
			\textbf{Latency (s)}                & x				& x				& x					& x							\\
			\textbf{Latency Speedup}            & 100\%			& x\%			& x\%				& x\%						\\
			\textbf{Total On-Chip Power (Watt)} & x				& x				& x					& x							\\
			\textbf{Power Efficiency}           & 100\%			& x\%			& x\%				& x\%						\\
			\textbf{Energy Cons./Image (Joule)} & x				& x				& x					& x							\\
			\textbf{Energy Efficiency}          & 100\%			& x\%			& x\%				& x\%						\\
			\textbf{Images/Joule}               & x				& x				& x					& x							\\
			\bottomrule
		\end{tabular}
	\end{table}
\end{frame}

% Todo: Replace graph
\begin{frame}{Final Performance}
	\centering
	\includegraphics[width=0.9\textwidth]{../Images/Results/Final-Results-charts.png}\\
\end{frame}

\setlength{\parskip}{\baselineskip}
\section{Conclusion}

\begin{frame}
	\huge Conclusions \& Future Work
\end{frame}

% Todo: Edit to your liking
\begin{frame}{Conclusions}
	\begin{itemize}
		\item Conclusion 1
		\item Conclusion 2
		\item Conclusion 3
	\end{itemize}
\end{frame}

% Todo: Edit to your liking
\begin{frame}{Future Work}
	\begin{itemize}
		\item Idea 1
		\item Idea 2
		\item Idea 3
	\end{itemize}
\end{frame}


% Todo: Uncomment these examples to see the official template
% \input{Chapters/example/introduction}
% \input{Chapters/example/datamodel}
% \input{Chapters/example/architecture}
% \input{Chapters/example/implementation}
% \input{Chapters/example/consistency}
% \input{Chapters/example/workload}

\setbeamercolor{background canvas}{bg=matblue}
\setbeamercolor{normal text}{fg=white}
\begin{frame}[plain, b]
	\centering
	\huge \textcolor{white}{Thank You!}\\
	\large \textcolor{white}{Any Questions?}
	\normalsize
	\vspace*{\fill}
\end{frame}

\end{document}
